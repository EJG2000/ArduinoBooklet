\chapter{Introduction to Robotics}
%------------------------------------------------

\section{What is Robotics?}
\par Robotics is an interdisciplinary field that integrates Computer Science and Engineering. It comprises design, construction, operation and use of robots. The field targets the creation of machines that can help and assist humans through blending aspects of mechanical engineering, electrical engineering, mechatronics, electronics etc
\par A Robot is an interactive machine capable of carrying out a complex series of actions automatically, especially one programmable by a computer, to reduce human risk in hazardous works. Thus, robotics is an interdisciplinary branch of engineering and science that deals with the design, construction, operation, and use of robots, as well as computer systems for their control, sensory feedback, and information processing to create an efficient robot. A robot can be thought as an combination of sensors, controllers and actuators. Sensor provide collection of information from environment, that are processed by controllers who generate signal to actuators to interact with environment.
\par Robots are put to use in different applications. Their usage is determined by the field they concentrate on. Robotics is a vast field. They may be manufactured or processed in different forms, shapes and utility. Joseph Engelberger, godfather of robotics, once said \textit{“I can't define a robot, but I know one when I see one.”}

\section{Evolution of Robotics}
\par The question of evolution of robotics take us to the ancient world and era of industrial revolution. Engineers were trying to develop machines that can handle dangerous tasks for automotive industry and defence. Robots they developed were meant to be a replica of human actions. 
\begin{table*}
    \centering
    \begin{tabular}{|c|p{14cm}|}
    \hline
        \textbf{\textit{Year}} & \hspace{30mm}\textbf{\textit{Major Events in History of Robotics}} \\ \hline
        
        \rowcolor[gray]{0.95} 1801 & Joseph Jacquard builds an automated loom that is controlled by a punch card. Punch cards are later used as an input method for some early 20th century computers. \\ \hline
        
        1921 & Writer Karel Capek introduced the term \textbf{'robot'} in his play 'Rossum's Universal Robots'. The word \textbf{'robot'} evolved from the word 'robota' meaning 'work'. \\ \hline
        
        \rowcolor[gray]{0.95} 1941 & Developed from the idea by Karel Capek, Science Fiction Writer Isaac Asimov first used the word \textbf{'robotics'} to describe the technology of robots. \\ \hline
        
        1942 & Isaac Asimov wrote the \textbf{'Three Laws of Robotics'} which is still considered as the foundational rules of robotic industry. \\ \hline
        
        \rowcolor[gray]{0.95} 1950 - 60 & The first ever industrial robot \textbf{Unimate} was designed by George Devol and Joe Engelberger. The 4,000-pound robot arm sequenced and stacked hot pieces of die-cast metal. \\ \hline
        
        1969 & Victor Scheinman designed a \textbf{Stanford arm} which eventually became a standard and is still influencing the design of robotic arms today. \\ \hline
        
        \rowcolor[gray]{0.95} 1974 & \textbf{IRB-6}, the first electric industrial robot that was controlled by a microcomputer. The major features were that it was programmable, had 16KB of RAM, and display four whole digits with LEDs. \\ \hline
        
        1986 & Honda introduced a \textbf{Robot Research Program} which came up with ideas that a robot should coexist and cooperate with human beings. \\ \hline
        
        \rowcolor[gray]{0.95} 1977 & The NASA \textbf{Pathfinder Mission} lands on Mars. Its robotic rover Sojourner broadcasts data from Martian surface. \\ \hline
        
        1999 & The Sony \textbf{Aibo} offers users a robotic pet dog with learning and communication potential. Aibo christens a revolutionary drive in consumer robotics and technology with more affordable iterations appearing throughout the next decade. \\ \hline
        
        \rowcolor[gray]{0.95} 2005 &  Honda’s \textbf{Advanced Step in Innovative Mobility (ASIMO)} humanoid robot is introduced. It could walk 1 mph, climb stairs and change its direction after detecting hazards. Using the camera mounted in its head, ASIMO could also recognize faces, gestures and the movements of multiple objects. \\ \hline
        
        2016 & \textbf{Sophia}, a social Humanoid robot developed by Hong Kong-based company Hanson Robotics. It is the first humanoid to receive citizenship to a country. It has the unique ability to connect and communicate with humans in an unforgettable way, showcasing our unparalleled robotics art and design, AI science, and engineering.\\ \hline
        
    \end{tabular}
\end{table*}